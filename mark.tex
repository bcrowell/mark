% a5 format with gutter, for lulu:
\documentclass[10pt,a5paper,twoside,twocolumn]{book}
\usepackage[bindingoffset=0.3in, left=0.5in, right=0.5in]{geometry} % https://tex.stackexchange.com/q/80520/6853

\usepackage{fontspec}
\setmainfont{DejaVu Serif} % This font looks like ass. But it makes greek work.

\usepackage{parskip} % https://tex.stackexchange.com/a/57/6853

\usepackage{fancyhdr}
\pagestyle{fancy}
%\fancyhf{}
\renewcommand{\headrulewidth}{0pt}
\fancyhead[RE]{ΛΟΓΟΣ \thechapter}
\fancyhead[LO]{CHAPTER \thechapter}
\fancyhead[LE,RO]{\thepage}

\usepackage{makeidx}
\usepackage{multicol}
\usepackage{enumitem} % for enumerate environments with letters
\usepackage{etoolbox} % supplies \ifstrempty
\usepackage{navigator}
\usepackage{xcolor}
\usepackage{url}
\usepackage{ifthen}
\usepackage{graphicx}
\graphicspath{ {./figs/} }
\newcommand{\fig}[1]{\includegraphics[width=\columnwidth]{#1}\label{fig:#1}}

\newcommand*\cleartoleftpage{%
  \ifodd\value{page}\hbox{}\clearpage\fi
}


\newcommand{\separator}{\hspace{0.27\textwidth}\noindent\makebox[\linewidth]{\resizebox{0.3333\linewidth}{1pt}{$\bullet$}}\bigskip}
% ... https://tex.stackexchange.com/a/130763/6853
\newenvironment{facing}{\cleartoleftpage}{\clearpage\pagebreak}
\newenvironment{help}{\pagebreak}{}
\newenvironment{helpsec}{\begin{minipage}[t]{\textwidth}\begin{multicols}{2}}{\end{multicols}\end{minipage}}
\newenvironment{vocab}{\begin{helpsec}}{\end{helpsec}}
\newenvironment{translation}{\separator\\\begin{helpsec}\footnotesize}{\end{helpsec}}

%----------------------------------------------------------------------------------

\begin{document}

%===================================================================================

\chapter{1}

\begin{facing}

1 Ἀρχὴ τοῦ εὐαγγελίου Ἰησοῦ Χριστοῦ.

2 Καθὼς γέγραπται ἐν τῷ Ἡσαΐᾳ τῷ προφήτῃ,
	«Ἰδοὺ ἀποστέλλω τὸν ἄγγελόν μου πρὸ προσώπου σου, 
	ὃς κατασκευάσει τὴν ὁδόν σου·
		3 φωνὴ βοῶντος ἐν τῇ ἐρήμῳ 
	Ἑτοιμάσατε τὴν ὁδὸν Κυρίου, 
	εὐθείας ποιεῖτε τὰς τρίβους αὐτοῦ!»

\fig{01-04} % https://commons.wikimedia.org/wiki/File:%27Saint_John_the_Baptist_Preaching%27_by_Luca_Giordano,_LACMA.JPG

4 ἐγένετο Ἰωάνης ὁ βαπτίζων ἐν τῇ ἐρήμῳ κηρύσσων βάπτισμα μετανοίας εἰς ἄφεσιν ἁμαρτιῶν. 5 καὶ ἐξεπορεύετο πρὸς αὐτὸν πᾶσα ἡ Ἰουδαία χώρα καὶ οἱ Ἱεροσολυμεῖται πάντες, καὶ ἐβαπτίζοντο ὑπ’ αὐτοῦ ἐν τῷ Ἰορδάνῃ ποταμῷ ἐξομολογούμενοι τὰς ἁμαρτίας αὐτῶν. 6 καὶ ἦν ὁ Ἰωάνης ἐνδεδυμένος τρίχας καμήλου καὶ ζώνην δερματίνην περὶ τὴν ὀσφὺν αὐτοῦ, καὶ ἔσθων ἀκρίδας καὶ μέλι ἄγριον. 7 καὶ ἐκήρυσσεν λέγων, «Ἔρχεται ὁ ἰσχυρότερός μου ὀπίσω μου, οὗ οὐκ εἰμὶ ἱκανὸς κύψας λῦσαι τὸν ἱμάντα τῶν ὑποδημάτων αὐτοῦ. 8 ἐγὼ ἐβάπτισα ὑμᾶς ὕδατι, αὐτὸς δὲ βαπτίσει ὑμᾶς Πνεύματι Ἁγίῳ.»

\fig{01-09} % https://commons.wikimedia.org/wiki/File:Joachim_Patinir_-_The_Baptism_of_Christ_-_Google_Art_Project_2.jpg

	9 Καὶ ἐγένετο ἐν ἐκείναις ταῖς ἡμέραις ἦλθεν Ἰησοῦς ἀπὸ Ναζαρὲτ τῆς Γαλιλαίας καὶ ἐβαπτίσθη εἰς τὸν Ἰορδάνην ὑπὸ Ἰωάνου. 10 καὶ εὐθὺς ἀναβαίνων ἐκ τοῦ ὕδατος εἶδεν σχιζομένους τοὺς οὐρανοὺς καὶ τὸ Πνεῦμα ὡς περιστερὰν καταβαῖνον εἰς αὐτόν· 
11 καὶ φωνὴ ἐγένετο ἐκ τῶν οὐρανῶν, «Σὺ εἶ ὁ Υἱός μου ὁ ἀγαπητός, ἐν σοὶ εὐδόκησα.»

%----------------------------------

\begin{help}

\begin{vocab}

\emph{rare}\\
ἀκρίς, -ίδος -- locust \\
κύπτω -- bend down\\
ὀσφῦς -- loin\\

\emph{common}\\
ἄγριος -- wild\\
βοάω -- shout\\
δοκέω -- seem, have an opinion\\
ἐγενόμην -- aor. of γίγνομαι\\
ἐνδύω -- wear\\
ἐρῆμος -- wilderness\\
ἐξομολογέομαι -- confess \\
εὐθύς -- immediately \\
ζώνη -- belt \\
ἦλθεν - aor.~of ἔρχομαι\\
ἰσχύς -- strength\\
κατασκευάζω -- build\\
κηρύσσω -- announce\\
λύω, ἔλῡσα -- loosen\\
μέλι, -ῐτος -- honey\\
ὀπίσω -- backwards; later\\
περιστερά -- dove\\
πορεύω - bring, go\\
τρῐχός -- gen. of θρίξ \\
τρίβος -- trail\\
\end{vocab}



\begin{translation}
1 The beginning of the Good News of Jesus Christ, the Son of God.

2 As it is written in the prophets,
``Behold, I send my messenger before your face,
who will prepare your way before you:
3 the voice of one crying in the wilderness,
`Make ready the way of the Lord!
Make his paths straight!'{}''

4 John came baptizing in the wilderness and preaching the baptism of repentance for forgiveness of sins. 5 All the country of Judea and all those of Jerusalem went out to him. They were baptized by him in the Jordan river, confessing their sins. 6 John was clothed with camel's hair and a leather belt around his waist. He ate locusts and wild honey. 7 He preached, saying, ``After me comes he who is mightier than I, the thong of whose sandals I am not worthy to stoop down and loosen. 8 I baptized you in water, but he will baptize you in the Holy Spirit.''

9 In those days, Jesus came from Nazareth of Galilee, and was baptized by John in the Jordan. 10 Immediately coming up from the water, he saw the heavens parting and the Spirit descending on him like a dove. 11 A voice came out of the sky, ``You are my beloved Son, in whom I am well pleased.'' 
\end{translation}

\end{help}
\end{facing}


%===================================================================================

\begin{facing}

\fig{01-12} % https://commons.wikimedia.org/wiki/File:Christ_in_the_Wilderness_-_Ivan_Kramskoy_-_Google_Cultural_Institute.jpg

12 Καὶ εὐθὺς τὸ Πνεῦμα αὐτὸν ἐκβάλλει εἰς τὴν ἔρημον. 13 καὶ ἦν ἐν τῇ ἐρήμῳ τεσσεράκοντα ἡμέρας πειραζόμενος ὑπὸ τοῦ Σατανᾶ, καὶ ἦν μετὰ τῶν θηρίων, καὶ οἱ ἄγγελοι διηκόνουν αὐτῷ. 

\fig{01-14} % https://commons.wikimedia.org/wiki/File:Duccio_di_Buoninsegna_036.jpg

	14 Καὶ μετὰ τὸ παραδοθῆναι τὸν Ἰωάνην ἦλθεν ὁ Ἰησοῦς εἰς τὴν Γαλιλαίαν κηρύσσων τὸ εὐαγγέλιον τοῦ Θεοῦ 15 καὶ λέγων ὅτι,
«Πεπλήρωται ὁ καιρὸς καὶ ἤγγικεν ἡ βασιλεία τοῦ Θεοῦ· μετανοεῖτε καὶ πιστεύετε ἐν τῷ εὐαγγελίῳ.»

16 Καὶ παράγων παρὰ τὴν θάλασσαν τῆς Γαλιλαίας εἶδεν Σίμωνα καὶ Ἀνδρέαν τὸν ἀδελφὸν Σίμωνος ἀμφιβάλλοντας ἐν τῇ θαλάσσῃ· ἦσαν γὰρ ἁλεεῖς. 
17 καὶ εἶπεν αὐτοῖς ὁ Ἰησοῦς, «Δεῦτε ὀπίσω μου, καὶ ποιήσω ὑμᾶς γενέσθαι ἁλεεῖς ἀνθρώπων.» 18 καὶ εὐθὺς ἀφέντες τὰ δίκτυα ἠκολούθησαν αὐτῷ.

 19 Καὶ προβὰς ὀλίγον εἶδεν Ἰάκωβον τὸν τοῦ Ζεβεδαίου καὶ Ἰωάνην τὸν ἀδελφὸν αὐτοῦ, καὶ αὐτοὺς ἐν τῷ πλοίῳ καταρτίζοντας τὰ δίκτυα, 20 καὶ εὐθὺς ἐκάλεσεν αὐτούς· καὶ ἀφέντες τὸν πατέρα αὐτῶν Ζεβεδαῖον ἐν τῷ πλοίῳ μετὰ τῶν μισθωτῶν ἀπῆλθον ὀπίσω αὐτοῦ.

\begin{help}
\begin{vocab}
\emph{common}\\
ἀφίημι -- drop, release\\
ἀκολουθέω -- follow\\
βαίνω -- walk, stand\\
παραδίδωμι -- hand over\\

\emph{rare}\\
ἁλιεύς -- fisherman\\
ἀμφιβάλλω -- throw around\\
δεῦτε -- Come here! (pl.)\\
διακονέω -- minister to, serve\\
δίκτῠον -- net\\
ἤγγικα -- perf. of ἐγγίζω, bring near\\
θηρίον -- wild animal\\
καταρτίζω -- fix, adjust\\
μισθωτής -- renter, tenant\\
πειράζω -- try, tempt\\
πλοίῳ -- boat\\
προβὰς -- form of προ+βαίνω\\
\end{vocab}
\begin{translation}
12 Immediately the Spirit drove him out into the wilderness. 13 He was there in the wilderness forty days, tempted by Satan. He was with the wild animals; and the angels were serving him.

14 Now after John was taken into custody, Jesus came into Galilee, preaching the Good News of God's Kingdom, 15 and saying, ``The time is fulfilled, and God's Kingdom is at hand! Repent, and believe in the Good News.''

16 Passing along by the sea of Galilee, he saw Simon and Andrew, the brother of Simon, casting a net into the sea, for they were fishermen. 17 Jesus said to them, ``Come after me, and I will make you into fishers for men.''
18 Immediately they left their nets, and followed him.

19 Going on a little further from there, he saw James the son of Zebedee, and John his brother, who were also in the boat mending the nets. 20 Immediately he called them, and they left their father, Zebedee, in the boat with the hired servants, and went after him. 
\end{translation}

\end{help}

\end{facing}

%===================================================================================

\begin{facing}

\fig{01-21} % https://commons.wikimedia.org/wiki/File:Christus_heilt_einen_Besessenen.jpg

21 Καὶ εἰσπορεύονται εἰς Καφαρναούμ· καὶ εὐθὺς τοῖς σάββασιν εἰσελθὼν εἰς τὴν συναγωγὴν ἐδίδασκεν. 22 καὶ ἐξεπλήσσοντο ἐπὶ τῇ διδαχῇ αὐτοῦ· ἦν γὰρ διδάσκων αὐτοὺς ὡς ἐξουσίαν ἔχων, καὶ οὐχ ὡς οἱ γραμματεῖς.

23 Καὶ εὐθὺς ἦν ἐν τῇ συναγωγῇ αὐτῶν ἄνθρωπος ἐν πνεύματι ἀκαθάρτῳ, καὶ ἀνέκραξεν, 
24 λέγων, «Τί ἡμῖν καὶ σοί, Ἰησοῦ Ναζαρηνέ; ἦλθες ἀπολέσαι ἡμᾶς. οἶδά σε τίς εἶ, ὁ Ἅγιος τοῦ Θεοῦ.»

25 καὶ ἐπετίμησεν αὐτῷ ὁ Ἰησοῦς, «Φιμώθητι καὶ ἔξελθε ἐξ αὐτοῦ.»

26 καὶ σπαράξαν αὐτὸν τὸ πνεῦμα τὸ ἀκάθαρτον καὶ φωνῆσαν φωνῇ μεγάλῃ ἐξῆλθεν ἐξ αὐτοῦ.
27 καὶ ἐθαμβήθησαν ἅπαντες, ὥστε συνζητεῖν αὐτοὺς λέγοντας, «Τί ἐστιν τοῦτο; διδαχὴ καινή κατ’ ἐξουσίαν· καὶ τοῖς πνεύμασι τοῖς ἀκαθάρτοις ἐπιτάσσει, 
καὶ ὑπακούουσιν αὐτῷ.»
28 καὶ ἐξῆλθεν ἡ ἀκοὴ αὐτοῦ εὐθὺς πανταχοῦ εἰς ὅλην τὴν περίχωρον τῆς Γαλιλαίας.

\fig{01-29} % https://upload.wikimedia.org/wikipedia/commons/3/3e/Christ_Healing_the_Mother_of_Simon_Peter%E2%80%99s_Wife_by_John_Bridges.jpg

29 Καὶ εὐθὺς ἐκ τῆς συναγωγῆς ἐξελθόντες ἦλθον εἰς τὴν οἰκίαν Σίμωνος καὶ Ἀνδρέου μετὰ Ἰακώβου καὶ Ἰωάνου. 30 ἡ δὲ πενθερὰ Σίμωνος κατέκειτο πυρέσσουσα, καὶ εὐθὺς λέγουσιν αὐτῷ περὶ αὐτῆς. 31 καὶ προσελθὼν ἤγειρεν αὐτὴν κρατήσας τῆς χειρός· καὶ ἀφῆκεν αὐτὴν ὁ πυρετός, καὶ διηκόνει αὐτοῖς. 

\begin{help}
\begin{vocab}
\emph{common}\\
ἀφίημι -- send forth, drop\\
ἀφήκω -- arrive, depart (imp.~of ἀφίημι)\\
ἀκάθαρτος -- unclean\\
ἀκούω -- hear\\
ἀκοή -- hearing, act of hearing, ear\\
ἀνακράζω -- cry out \\
ἐκπλήσσω -- astound, knock out\\
ἐπιτιμάω -- honor, rebuke\\
θαῦμα -- wonder, astonishment\\
οἰκία -- house\\
πανταχοῦ -- everywhere\\
πῦρ -- fire\\
ζητέω -- seek \\
ὑπακούω -- obey\\

\emph{rare}\\
διάκονος -- servant\\
διακονέω -- serve\\
θαμβέω -- to wonder, be amazed\\
φιμόω -- muzzle\\
πενθερά -- mother in law\\
πυρέσσω -- to have a fever\\
σπαράσσω -- rend, convulse\\
συνζητέω = συν+ζητέω
\end{vocab}
\begin{translation}
They went into Capernaum, and immediately on the Sabbath day he entered into the synagogue and taught. 22 They were astonished at his teaching, for he taught them as having authority, and not as the scribes. 23 Immediately there was in their synagogue a man with an unclean spirit, and he cried out, 24 saying, ``Ha! What do we have to do with you, Jesus, you Nazarene? Have you come to destroy us? I know you who you are: the Holy One of God!''

25 Jesus rebuked him, saying, ``Be quiet, and come out of him!''

26 The unclean spirit, convulsing him and crying with a loud voice, came out of him. 27 They were all amazed, so that they questioned among themselves, saying, ``What is this? A new teaching? For with authority he commands even the unclean spirits, and they obey him!'' 28 The report of him went out immediately everywhere into all the region of Galilee and its surrounding area.

29 Immediately, when they had come out of the synagogue, they came into the house of Simon and Andrew, with James and John. 30 Now Simon's wife's mother lay sick with a fever, and immediately they told him about her. 31 He came and took her by the hand and raised her up. The fever left her immediately, and she served them. 
\end{translation}
\end{help}
\end{facing}

%===================================================================================

\begin{facing}

32 Ὀψίας δὲ γενομένης, ὅτε ἔδυσεν ὁ ἥλιος, ἔφερον πρὸς αὐτὸν πάντας τοὺς κακῶς ἔχοντας καὶ τοὺς δαιμονιζομένους· 33 καὶ ἦν ὅλη ἡ πόλις ἐπισυνηγμένη πρὸς τὴν θύραν.
 34 καὶ ἐθεράπευσεν πολλοὺς κακῶς ἔχοντας ποικίλαις νόσοις, καὶ δαιμόνια πολλὰ ἐξέβαλεν, καὶ οὐκ ἤφιεν λαλεῖν τὰ δαιμόνια, ὅτι ᾔδεισαν αὐτόν. 

\fig{01-34} % https://commons.wikimedia.org/wiki/File:Jesus_sana_a_los_enfermos_-ig_Nstra_Sra_Angeles_2der_fRF2.2.jpg


\fig{01-35} % https://commons.wikimedia.org/wiki/File:Brooklyn_Museum_-_Christ_Retreats_to_the_Mountain_at_Night_(J%C3%A9sus_se_retira_la_nuit_sur_la_montagne)_-_James_Tissot.jpg

35 Καὶ πρωῒ ἔννυχα λίαν ἀναστὰς ἐξῆλθεν καὶ ἀπῆλθεν εἰς ἔρημον τόπον, κἀκεῖ προσηύχετο.

36 καὶ κατεδίωξεν αὐτὸν Σίμων καὶ οἱ μετ’ αὐτοῦ, 37 καὶ εὗρον αὐτὸν καὶ λέγουσιν αὐτῷ ὅτι Πάντες ζητοῦσίν σε. 38 καὶ λέγει αὐτοῖς· Ἄγωμεν ἀλλαχοῦ εἰς τὰς ἐχομένας κωμοπόλεις, ἵνα καὶ ἐκεῖ κηρύξω· εἰς τοῦτο γὰρ ἐξῆλθον. 39 καὶ ἦλθεν κηρύσσων εἰς τὰς συναγωγὰς αὐτῶν εἰς ὅλην τὴν Γαλιλαίαν καὶ τὰ δαιμόνια ἐκβάλλων.

\begin{help}
\begin{vocab}
\emph{common}\\
ἀναστὰς -- aor. of ᾰ̓νᾰ + ῐ̔́στημῐ \\
διώκω -- chase \\
θύρα -- door\\
οἶδα -- know\\
ᾔδεα -- pluperfect of οἶδα, used as imperfect\\
πρωΐ -- early in the day\\

\emph{rare}\\
κἀκεῖ = καὶ ἐκεῖ \\
ὀψία -- evening\\
\end{vocab}

\begin{translation}
32 At evening, when the sun had set, they brought to him all who were sick and those who were possessed by demons. 33 All the city was gathered together at the door. 34 He healed many who were sick with various diseases and cast out many demons. He didn't allow the demons to speak, because they knew him.

35 Early in the morning, while it was still dark, he rose up and went out, and departed into a deserted place, and prayed there.

36 Simon and those who were with him searched for him. 37 They found him and told him, ``Everyone is looking for you.''
38 He said to them, ``Let's go elsewhere into the next towns, that I may preach there also, because I came out for this reason.'' 39 He went into their synagogues throughout all Galilee, preaching and casting out demons. 

\end{translation}
\end{help}
\end{facing}

%===================================================================================

\begin{facing}

\fig{01-40} % https://commons.wikimedia.org/wiki/File:Iscelenie_prakajennogo.jpg

	40 Καὶ ἔρχεται πρὸς αὐτὸν λεπρὸς παρακαλῶν αὐτὸν καὶ γονυπετῶν λέγων αὐτῷ ὅτι, «Ἐὰν θέλῃς δύνασαί με καθαρίσαι.»


41 καὶ σπλαγχνισθεὶς ἐκτείνας τὴν χεῖρα αὐτοῦ ἥψατο καὶ λέγει, «αὐτῷ Θέλω, καθαρίσθητι.» 42 καὶ εὐθὺς ἀπῆλθεν ἀπ’ αὐτοῦ ἡ λέπρα, καὶ ἐκαθερίσθη. 43 καὶ ἐμβριμησάμενος αὐτῷ εὐθὺς ἐξέβαλεν αὐτόν, 44 καὶ λέγει αὐτῷ, «Ὅρα μηδενὶ μηδὲν εἴπῃς, ἀλλὰ ὕπαγε σεαυτὸν δεῖξον τῷ ἱερεῖ καὶ προσένεγκε περὶ τοῦ καθαρισμοῦ σου ἃ προσέταξεν Μωϋσῆς εἰς μαρτύριον αὐτοῖς.»

45 ὁ δὲ ἐξελθὼν ἤρξατο κηρύσσειν πολλὰ καὶ διαφημίζειν τὸν λόγον, ὥστε μηκέτι αὐτὸν δύνασθαι φανερῶς εἰς πόλιν εἰσελθεῖν, ἀλλ’ ἔξω ἐπ’ ἐρήμοις τόποις ἦν· καὶ ἤρχοντο πρὸς αὐτὸν πάντοθεν.

\begin{help}
\begin{vocab}
\emph{common}\\
γόνυ -- knee\\
ὅρα -- here, lit.~``see that''\\
προστάσσω -- post troops or an ordinance\\
φανερός -- visible\\
τείνω -- stretch\\

\emph{rare}\\
γονυπετέω -- kneel\\
διαφημίζω -- from φημί: spread word around\\
ἐμβριμάομαι -- snort like a horse; admonish\\
ἥψα -- aor. of ἅπτω\\
λεπρὸς -- leper\\
προσέταξεν -- ordinance\\

\end{vocab}
\begin{translation}
 40 A leper came to him, begging him, kneeling down to him, and saying to him, ``If you want to, you can make me clean.''

41 Being moved with compassion, he stretched out his hand, and touched him, and said to him, ``I want to. Be made clean.'' 42 When he had said this, immediately the leprosy departed from him and he was made clean. 43 He strictly warned him and immediately sent him out, 44 and said to him, ``See that you say nothing to anybody, but go show yourself to the priest and offer for your cleansing the things which Moses commanded, for a testimony to them.''

45 But he went out, and began to proclaim it much, and to spread about the matter, so that Jesus could no more openly enter into a city, but was outside in desert places. People came to him from everywhere. 

\end{translation}
\end{help}
\end{facing}

%===================================================================================
%===================================================================================
%                                2
%===================================================================================
%===================================================================================

\chapter{2}

\begin{facing}

\fig{02-04} % /home/bcrowell/Documents/writing/jesus/figs/healing-the-paralytic.jpg

	1 Καὶ εἰσελθὼν πάλιν εἰς Καφαρναοὺμ δι’ ἡμερῶν ἠκούσθη ὅτι ἐν οἴκῳ ἐστίν. 2 καὶ συνήχθησαν πολλοὶ, ὥστε μηκέτι χωρεῖν μηδὲ τὰ πρὸς τὴν θύραν, καὶ ἐλάλει αὐτοῖς τὸν λόγον. 3 καὶ ἔρχονται φέροντες πρὸς αὐτὸν παραλυτικὸν αἰρόμενον ὑπὸ τεσσάρων. 4 καὶ μὴ δυνάμενοι προσενέγκαι αὐτῷ διὰ τὸν ὄχλον ἀπεστέγασαν τὴν στέγην ὅπου ἦν, καὶ ἐξορύξαντες χαλῶσι τὸν κράβαττον ὅπου ὁ παραλυτικὸς κατέκειτο. 
5 καὶ ἰδὼν ὁ Ἰησοῦς τὴν πίστιν αὐτῶν λέγει τῷ παραλυτικῷ· «Τέκνον, ἀφίενταί σου αἱ ἁμαρτίαι.»

 6 ἦσαν δέ τινες τῶν γραμματέων ἐκεῖ καθήμενοι καὶ διαλογιζόμενοι ἐν ταῖς καρδίαις αὐτῶν, «7 Τί οὗτος οὕτως λαλεῖ; βλασφημεῖ· τίς δύναται ἀφιέναι ἁμαρτίας εἰ μὴ εἷς ὁ Θεός;»

8 καὶ εὐθὺς ἐπιγνοὺς ὁ Ἰησοῦς τῷ πνεύματι αὐτοῦ ὅτι οὕτως διαλογίζονται ἐν ἑαυτοῖς, λέγει αὐτοῖς, «Τί ταῦτα διαλογίζεσθε ἐν ταῖς καρδίαις ὑμῶν; 
9 τί ἐστιν εὐκοπώτερον, εἰπεῖν τῷ παραλυτικῷ, ``Ἀφίενταί σου αἱ ἁμαρτίαι,'' ἢ εἰπεῖν, ``Ἔγειρε καὶ ἆρον τὸν κράβαττόν σου καὶ περιπάτει;''

10 ἵνα δὲ εἰδῆτε ὅτι ἐξουσίαν ἔχει ὁ Υἱὸς τοῦ ἀνθρώπου ἀφιέναι ἁμαρτίας ἐπὶ τῆς γῆς,» — λέγει τῷ παραλυτικῷ 11 «Σοὶ λέγω, ἔγειρε ἆρον τὸν κράβαττόν σου καὶ ὕπαγε εἰς τὸν οἶκόν σου.»

\fig{02-06} % https://commons.wikimedia.org/wiki/File:Duccio_di_Buoninsegna_-_Christ_Accused_by_the_Pharisees_(detail)_-_WGA06802.jpg



\begin{help}
\begin{vocab}

\emph{common}\\
στέγη -- roof \\
ὀρύσσω -- dig \\
πίστις -- faith \\

\emph{rare}\\
ἐπιγιγνώσκω -- recogize\\
εὔκοπος -- easily (=εὔπορος)\\
κράββατος -- bed, mat\\

\emph{forms and compounds}\\
ἀποστεγάζω = ἀπο + στεγ + άζω, uncover\\
ἐξορύσσω= ἐξ + ὀρύσσω \\
προσενέγκαι -- aor. optative of προσ + φέρω \\
συνήχθησαν -- aor. passive of συν + άγω \\
\end{vocab}
\begin{translation}
 1 When he entered again into Capernaum after some days, it was heard that he was at home. 2 Immediately many were gathered together, so that there was no more room, not even around the door; and he spoke the word to them. 3 Four people came, carrying a paralytic to him. 4 When they could not come near to him for the crowd, they removed the roof where he was. When they had broken it up, they let down the mat that the paralytic was lying on. 5 Jesus, seeing their faith, said to the paralytic, ``Son, your sins are forgiven you.''

6 But there were some of the scribes sitting there and reasoning in their hearts, 7 ``Why does this man speak blasphemies like that? Who can forgive sins but God alone?''

8 Immediately Jesus, perceiving in his spirit that they so reasoned within themselves, said to them, ``Why do you reason these things in your hearts? 9  Which is easier, to tell the paralytic, `Your sins are forgiven;' or to say, `Arise, and take up your bed, and walk?' 10  But that you may know that the Son of Man has authority on earth to forgive sins''--he said to the paralytic-- 11  ``I tell you, arise, take up your mat, and go to your house.'' 
\end{translation}
\end{help}
\end{facing}

%===================================================================================

\begin{facing}
 12 καὶ ἠγέρθη καὶ εὐθὺς ἄρας τὸν κράβαττον ἐξῆλθεν ἔμπροσθεν πάντων, ὥστε ἐξίστασθαι πάντας καὶ δοξάζειν τὸν Θεὸν λέγοντας ὅτι,
«Οὕτως οὐδέποτε εἴδαμεν.»

\fig{02-13} % https://commons.wikimedia.org/wiki/File:Brooklyn_Museum_-_Jesus_Teaches_the_People_by_the_Sea_(J%C3%A9sus_enseigne_le_peuple_pr%C3%A8s_de_la_mer)_-_James_Tissot_-_overall.jpg

	13 Καὶ ἐξῆλθεν πάλιν παρὰ τὴν θάλασσαν· καὶ πᾶς ὁ ὄχλος ἤρχετο πρὸς αὐτόν, καὶ ἐδίδασκεν αὐτούς. 

\fig{02-14} % https://commons.wikimedia.org/wiki/File:Hole_jesus_calls_levi.jpg

14 καὶ παράγων εἶδεν Λευεὶν τὸν τοῦ Ἀλφαίου καθήμενον ἐπὶ \pagebreak
τὸ τελώνιον, «καὶ λέγει αὐτῷ Ἀκολούθει μοι. καὶ ἀναστὰς ἠκολούθησεν αὐτῷ.»

\fig{02-15} % https://commons.wikimedia.org/wiki/File:Brooklyn_Museum_-_The_Meal_in_the_House_of_Matthew_(Le_repas_chez_Mathieu)_-_James_Tissot.jpg


15 Καὶ γίνεται κατακεῖσθαι αὐτὸν ἐν τῇ οἰκίᾳ αὐτοῦ, καὶ πολλοὶ τελῶναι καὶ ἁμαρτωλοὶ συνανέκειντο τῷ Ἰησοῦ καὶ τοῖς μαθηταῖς αὐτοῦ, ἦσαν γὰρ πολλοὶ καὶ ἠκολούθουν αὐτῷ. 16 καὶ οἱ γραμματεῖς τῶν Φαρισαίων ἰδόντες ὅτι ἐσθίει μετὰ τῶν ἁμαρτωλῶν καὶ τελωνῶν, ἔλεγον τοῖς μαθηταῖς αὐτοῦ 
«Ὅτι, μετὰ τῶν τελωνῶν καὶ ἁμαρτωλῶν ἐσθίει; 17 καὶ ἀκούσας ὁ Ἰησοῦς λέγει αὐτοῖς Οὐ χρείαν ἔχουσιν οἱ ἰσχύοντες ἰατροῦ ἀλλ’ οἱ κακῶς ἔχοντες· οὐκ ἦλθον καλέσαι δικαίους ἀλλὰ ἁμαρτωλούς.»


\begin{help}
\begin{vocab}
\emph{common}\\
κεῖμαι -- lie\\
τέλος -- end, power, tax\\
ὅτι -- in 2:16, ``why...''\\

\pagebreak

\emph{forms and compounds}\\
ἀνίστημι = aor. of ἀνίστημι\\
ἔμπροσθεν = ἐν + πρός + θεν \\
ἠγέρθη -- aor. of ἐγείρω\\
κατάκειμαι -- recline\\
τελώνης -- tax collector\\
\end{vocab}
\begin{translation}
 12 He arose, and immediately took up the mat and went out in front of them all, so that they were all amazed and glorified God, saying, ``We never saw anything like this!''
13 He went out again by the seaside. All the multitude came to him, and he taught them. 14 As he passed by, he saw Levi the son of Alphaeus sitting at the tax office. He said to him, ``Follow me.'' And he arose and followed him.
15 He was reclining at the table in his house, and many tax collectors and sinners sat down with Jesus and his disciples, for there were many, and they followed him. 16 The scribes and the Pharisees, when they saw that he was eating with the sinners and tax collectors, said to his disciples, ``Why is it that he eats and drinks with tax collectors and sinners?''
17 When Jesus heard it, he said to them, ``Those who are healthy have no need for a physician, but those who are sick. I came not to call the righteous, but sinners to repentance.'' 
\end{translation}
\end{help}
\end{facing}

%===================================================================================

18 Καὶ ἦσαν οἱ μαθηταὶ Ἰωάνου καὶ οἱ Φαρισαῖοι νηστεύοντες. καὶ ἔρχονται καὶ λέγουσιν αὐτῷ Διὰ τί οἱ μαθηταὶ Ἰωάνου καὶ οἱ μαθηταὶ τῶν Φαρισαίων νηστεύουσιν, οἱ δὲ σοὶ μαθηταὶ οὐ νηστεύουσιν; 

\fig{02-19} % https://commons.wikimedia.org/wiki/File:Paolo_Veronese_-_Die_Hochzeit_zu_Kana_-_ca1570.jpeg

19 καὶ εἶπεν αὐτοῖς ὁ Ἰησοῦς Μὴ δύνανται οἱ υἱοὶ τοῦ νυμφῶνος ἐν ᾧ ὁ νυμφίος μετ’ αὐτῶν ἐστιν νηστεύειν; ὅσον χρόνον ἔχουσιν τὸν νυμφίον μετ’ αὐτῶν, οὐ δύνανται νηστεύειν. 20 ἐλεύσονται δὲ ἡμέραι ὅταν ἀπαρθῇ ἀπ’ αὐτῶν ὁ νυμφίος, καὶ τότε νηστεύσουσιν ἐν ἐκείνῃ τῇ ἡμέρᾳ. 21 Οὐδεὶς ἐπίβλημα ῥάκους ἀγνάφου ἐπιράπτει ἐπὶ ἱμάτιον παλαιόν· εἰ δὲ μή, αἴρει τὸ πλήρωμα ἀπ’ αὐτοῦ τὸ καινὸν τοῦ παλαιοῦ, καὶ χεῖρον σχίσμα γίνεται. 22 καὶ οὐδεὶς βάλλει οἶνον νέον εἰς ἀσκοὺς παλαιούς· εἰ δὲ μή, ῥήξει ὁ οἶνος τοὺς ἀσκούς, καὶ ὁ οἶνος ἀπόλλυται καὶ οἱ ἀσκοί. ἀλλὰ οἶνον νέον εἰς ἀσκοὺς καινούς.

\fig{02-23} % https://commons.wikimedia.org/wiki/File:The_Bible_panorama,_or_The_Holy_Scriptures_in_picture_and_story_(1891)_(14785018525).jpg

	23 Καὶ ἐγένετο αὐτὸν ἐν τοῖς σάββασιν παραπορεύεσθαι διὰ τῶν σπορίμων, καὶ οἱ μαθηταὶ αὐτοῦ ἤρξαντο ὁδὸν ποιεῖν τίλλοντες τοὺς στάχυας. 24 καὶ οἱ Φαρισαῖοι ἔλεγον αὐτῷ Ἴδε τί ποιοῦσιν τοῖς σάββασιν ὃ οὐκ ἔξεστιν; 



\begin{facing}
\begin{help}
\begin{vocab}
\emph{common}\\

\end{vocab}
\begin{translation}
 18 John's disciples and the Pharisees were fasting, and they came and asked him, ``Why do John's disciples and the disciples of the Pharisees fast, but your disciples don't fast?''
19 Jesus said to them, ``Can the groomsmen fast while the bridegroom is with them? As long as they have the bridegroom with them, they can't fast. 20  But the days will come when the bridegroom will be taken away from them, and then they will fast in that day. 21  No one sews a piece of unshrunk cloth on an old garment, or else the patch shrinks and the new tears away from the old, and a worse hole is made. 22  No one puts new wine into old wineskins; or else the new wine will burst the skins, and the wine pours out, and the skins will be destroyed; but they put new wine into fresh wineskins.''
23 He was going on the Sabbath day through the grain fields; and his disciples began, as they went, to pluck the ears of grain. 24 The Pharisees said to him, ``Behold, why do they do that which is not lawful on the Sabbath day?'' 

\end{translation}
\end{help}
\end{facing}

%===================================================================================

\begin{facing}
25 καὶ λέγει αὐτοῖς· Οὐδέποτε ἀνέγνωτε τί ἐποίησεν Δαυείδ, ὅτε χρείαν ἔσχεν καὶ ἐπείνασεν αὐτὸς καὶ οἱ μετ’ αὐτοῦ; 26 πῶς εἰσῆλθεν εἰς τὸν οἶκον τοῦ Θεοῦ ἐπὶ Ἀβιαθὰρ ἀρχιερέως καὶ τοὺς ἄρτους τῆς προθέσεως ἔφαγεν, οὓς οὐκ ἔξεστιν φαγεῖν εἰ μὴ τοὺς ἱερεῖς, καὶ ἔδωκεν καὶ τοῖς σὺν αὐτῷ οὖσιν; 

27 καὶ ἔλεγεν αὐτοῖς Τὸ σάββατον διὰ τὸν ἄνθρωπον ἐγένετο, καὶ οὐχ ὁ ἄνθρωπος διὰ τὸ σάββατον· 28 ὥστε κύριός ἐστιν ὁ Υἱὸς τοῦ ἀνθρώπου καὶ τοῦ σαββάτου.

\begin{help}
\begin{vocab}
\emph{common}\\

\end{vocab}
\begin{translation}
 25 He said to them, ``Did you never read what David did when he had need and was hungry--he, and those who were with him? 26  How he entered into God's house at the time of Abiathar the high priest, and ate the show bread, which is not lawful to eat except for the priests, and gave also to those who were with him?''
27 He said to them, ``The Sabbath was made for man, not man for the Sabbath. 28  Therefore the Son of Man is lord even of the Sabbath.''
\end{translation}
\end{help}
\end{facing}


\sloppypar


\end{document}
