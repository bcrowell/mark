% a5 format with gutter, for lulu:
\documentclass[10pt,a5paper,twoside,twocolumn]{book}
\usepackage[bindingoffset=0.3in, left=0.5in, right=0.5in]{geometry} % https://tex.stackexchange.com/q/80520/6853

\usepackage{fontspec}
\setmainfont{DejaVu Serif} % This font looks like ass. But it makes greek work.

\usepackage{parskip} % https://tex.stackexchange.com/a/57/6853

\usepackage{makeidx}
\usepackage{multicol}
\usepackage{enumitem} % for enumerate environments with letters
\usepackage{etoolbox} % supplies \ifstrempty
\usepackage{navigator}
\usepackage{xcolor}
\usepackage{url}
\usepackage{ifthen}
\usepackage{graphicx}
\graphicspath{ {./figs/} }
\newcommand{\fig}[1]{\includegraphics[width=\columnwidth]{#1}\label{fig:#1}}

\newcommand*\cleartoleftpage{%
  \ifodd\value{page}\hbox{}\clearpage\fi
}


\newcommand{\separator}{\hspace{0.27\textwidth}\noindent\makebox[\linewidth]{\resizebox{0.3333\linewidth}{1pt}{$\bullet$}}\bigskip}
% ... https://tex.stackexchange.com/a/130763/6853
\newenvironment{facing}{\cleartoleftpage}{\clearpage\pagebreak}
\newenvironment{help}{\pagebreak}{}
\newenvironment{helpsec}{\begin{minipage}[t]{\textwidth}\begin{multicols}{2}}{\end{multicols}\end{minipage}}
\newenvironment{vocab}{\begin{helpsec}}{\end{helpsec}}
\newenvironment{translation}{\separator\\\begin{helpsec}\footnotesize}{\end{helpsec}}

%----------------------------------------------------------------------------------

\begin{document}

%===================================================================================

\begin{facing}

1 Ἀρχὴ τοῦ εὐαγγελίου Ἰησοῦ Χριστοῦ.

2 Καθὼς γέγραπται ἐν τῷ Ἡσαΐᾳ τῷ προφήτῃ,
	«Ἰδοὺ ἀποστέλλω τὸν ἄγγελόν μου πρὸ προσώπου σου, 
	ὃς κατασκευάσει τὴν ὁδόν σου·
		3 φωνὴ βοῶντος ἐν τῇ ἐρήμῳ 
	Ἑτοιμάσατε τὴν ὁδὸν Κυρίου, 
	εὐθείας ποιεῖτε τὰς τρίβους αὐτοῦ!»

\fig{01-04} % https://commons.wikimedia.org/wiki/File:%27Saint_John_the_Baptist_Preaching%27_by_Luca_Giordano,_LACMA.JPG

4 ἐγένετο Ἰωάνης ὁ βαπτίζων ἐν τῇ ἐρήμῳ κηρύσσων βάπτισμα μετανοίας εἰς ἄφεσιν ἁμαρτιῶν. 5 καὶ ἐξεπορεύετο πρὸς αὐτὸν πᾶσα ἡ Ἰουδαία χώρα καὶ οἱ Ἱεροσολυμεῖται πάντες, καὶ ἐβαπτίζοντο ὑπ’ αὐτοῦ ἐν τῷ Ἰορδάνῃ ποταμῷ ἐξομολογούμενοι τὰς ἁμαρτίας αὐτῶν. 6 καὶ ἦν ὁ Ἰωάνης ἐνδεδυμένος τρίχας καμήλου καὶ ζώνην δερματίνην περὶ τὴν ὀσφὺν αὐτοῦ, καὶ ἔσθων ἀκρίδας καὶ μέλι ἄγριον. 7 καὶ ἐκήρυσσεν λέγων, «Ἔρχεται ὁ ἰσχυρότερός μου ὀπίσω μου, οὗ οὐκ εἰμὶ ἱκανὸς κύψας λῦσαι τὸν ἱμάντα τῶν ὑποδημάτων αὐτοῦ. 8 ἐγὼ ἐβάπτισα ὑμᾶς ὕδατι, αὐτὸς δὲ βαπτίσει ὑμᾶς Πνεύματι Ἁγίῳ.»

\fig{01-09} % https://commons.wikimedia.org/wiki/File:Joachim_Patinir_-_The_Baptism_of_Christ_-_Google_Art_Project_2.jpg

	9 Καὶ ἐγένετο ἐν ἐκείναις ταῖς ἡμέραις ἦλθεν Ἰησοῦς ἀπὸ Ναζαρὲτ τῆς Γαλιλαίας καὶ ἐβαπτίσθη εἰς τὸν Ἰορδάνην ὑπὸ Ἰωάνου. 10 καὶ εὐθὺς ἀναβαίνων ἐκ τοῦ ὕδατος εἶδεν σχιζομένους τοὺς οὐρανοὺς καὶ τὸ Πνεῦμα ὡς περιστερὰν καταβαῖνον εἰς αὐτόν· 
11 καὶ φωνὴ ἐγένετο ἐκ τῶν οὐρανῶν, «Σὺ εἶ ὁ Υἱός μου ὁ ἀγαπητός, ἐν σοὶ εὐδόκησα.»

%----------------------------------

\begin{help}

\begin{vocab}

\emph{rare}\\
ἀκρίς, -ίδος -- locust \\
κύπτω -- bend down\\
ὀσφῦς -- loin\\

\emph{common}\\
ἄγριος -- wild\\
βοάω -- shout\\
δοκέω -- seem, have an opinion\\
ἐγενόμην -- aor. of γίγνομαι\\
ἐνδύω -- wear\\
ἐρημία -- wilderness\\
ἐξομολογέομαι -- confess \\
εὐθύς -- immediately \\
ζώνη -- belt \\
ἰσχύς -- strength\\
κατασκευάζω -- build\\
κηρύσσω -- announce\\
λύω, ἔλῡσα -- loosen\\
μέλι, -ῐτος -- honey\\
ὀπίσω -- backwards; later\\
τρῐχός -- gen. of θρίξ \\
τρίβος -- trail\\
\end{vocab}



\begin{translation}
1 The beginning of the Good News of Jesus Christ, the Son of God.

2 As it is written in the prophets,
``Behold, I send my messenger before your face,
who will prepare your way before you:
3 the voice of one crying in the wilderness,
`Make ready the way of the Lord!
Make his paths straight!'{}''

4 John came baptizing in the wilderness and preaching the baptism of repentance for forgiveness of sins. 5 All the country of Judea and all those of Jerusalem went out to him. They were baptized by him in the Jordan river, confessing their sins. 6 John was clothed with camel's hair and a leather belt around his waist. He ate locusts and wild honey. 7 He preached, saying, ``After me comes he who is mightier than I, the thong of whose sandals I am not worthy to stoop down and loosen. 8 I baptized you in water, but he will baptize you in the Holy Spirit.''

9 In those days, Jesus came from Nazareth of Galilee, and was baptized by John in the Jordan. 10 Immediately coming up from the water, he saw the heavens parting and the Spirit descending on him like a dove. 11 A voice came out of the sky, ``You are my beloved Son, in whom I am well pleased.'' 
\end{translation}

\end{help}
\end{facing}


%===================================================================================

\fig{01-12} % https://commons.wikimedia.org/wiki/File:Christ_in_the_Wilderness_-_Ivan_Kramskoy_-_Google_Cultural_Institute.jpg

12 Καὶ εὐθὺς τὸ Πνεῦμα αὐτὸν ἐκβάλλει εἰς τὴν ἔρημον. 13 καὶ ἦν ἐν τῇ ἐρήμῳ τεσσεράκοντα ἡμέρας πειραζόμενος ὑπὸ τοῦ Σατανᾶ, καὶ ἦν μετὰ τῶν θηρίων, καὶ οἱ ἄγγελοι διηκόνουν αὐτῷ. 

\fig{01-14} % https://commons.wikimedia.org/wiki/File:Duccio_di_Buoninsegna_036.jpg

	14 Καὶ μετὰ τὸ παραδοθῆναι τὸν Ἰωάνην ἦλθεν ὁ Ἰησοῦς εἰς τὴν Γαλιλαίαν κηρύσσων τὸ εὐαγγέλιον τοῦ Θεοῦ 15 καὶ λέγων ὅτι Πεπλήρωται ὁ καιρὸς καὶ ἤγγικεν ἡ βασιλεία τοῦ Θεοῦ· μετανοεῖτε καὶ πιστεύετε ἐν τῷ εὐαγγελίῳ. 

16 Καὶ παράγων παρὰ τὴν θάλασσαν τῆς Γαλιλαίας εἶδεν Σίμωνα καὶ Ἀνδρέαν τὸν ἀδελφὸν Σίμωνος ἀμφιβάλλοντας ἐν τῇ θαλάσσῃ· ἦσαν γὰρ ἁλεεῖς. 17 καὶ εἶπεν αὐτοῖς ὁ Ἰησοῦς Δεῦτε ὀπίσω μου, καὶ ποιήσω ὑμᾶς γενέσθαι ἁλεεῖς ἀνθρώπων. 18 καὶ εὐθὺς ἀφέντες τὰ δίκτυα ἠκολούθησαν αὐτῷ.

 19 Καὶ προβὰς ὀλίγον εἶδεν Ἰάκωβον τὸν τοῦ Ζεβεδαίου καὶ Ἰωάνην τὸν ἀδελφὸν αὐτοῦ, καὶ αὐτοὺς ἐν τῷ πλοίῳ καταρτίζοντας τὰ δίκτυα, 20 καὶ εὐθὺς ἐκάλεσεν αὐτούς· καὶ ἀφέντες τὸν πατέρα αὐτῶν Ζεβεδαῖον ἐν τῷ πλοίῳ μετὰ τῶν μισθωτῶν ἀπῆλθον ὀπίσω αὐτοῦ.

\fig{01-21} % https://commons.wikimedia.org/wiki/File:Christus_heilt_einen_Besessenen.jpg

	21 Καὶ εἰσπορεύονται εἰς Καφαρναούμ· καὶ εὐθὺς τοῖς σάββασιν εἰσελθὼν εἰς τὴν συναγωγὴν ἐδίδασκεν. 22 καὶ ἐξεπλήσσοντο ἐπὶ τῇ διδαχῇ αὐτοῦ· ἦν γὰρ διδάσκων αὐτοὺς ὡς ἐξουσίαν ἔχων, καὶ οὐχ ὡς οἱ γραμματεῖς. 23 Καὶ εὐθὺς ἦν ἐν τῇ συναγωγῇ αὐτῶν ἄνθρωπος ἐν πνεύματι ἀκαθάρτῳ, καὶ ἀνέκραξεν 24 λέγων Τί ἡμῖν καὶ σοί, Ἰησοῦ Ναζαρηνέ; ἦλθες ἀπολέσαι ἡμᾶς. οἶδά σε τίς εἶ, ὁ Ἅγιος τοῦ Θεοῦ. 

25 καὶ ἐπετίμησεν αὐτῷ ὁ Ἰησοῦς Φιμώθητι καὶ ἔξελθε ἐξ αὐτοῦ.

 26 καὶ σπαράξαν αὐτὸν τὸ πνεῦμα τὸ ἀκάθαρτον καὶ φωνῆσαν φωνῇ μεγάλῃ ἐξῆλθεν ἐξ αὐτοῦ. 27 καὶ ἐθαμβήθησαν ἅπαντες, ὥστε συνζητεῖν αὐτοὺς λέγοντας Τί ἐστιν τοῦτο; διδαχὴ καινή κατ’ ἐξουσίαν· καὶ τοῖς πνεύμασι τοῖς ἀκαθάρτοις ἐπιτάσσει, καὶ ὑπακούουσιν αὐτῷ. 28 καὶ ἐξῆλθεν ἡ ἀκοὴ αὐτοῦ εὐθὺς πανταχοῦ εἰς ὅλην τὴν περίχωρον τῆς Γαλιλαίας.

\fig{01-29} % https://upload.wikimedia.org/wikipedia/commons/3/3e/Christ_Healing_the_Mother_of_Simon_Peter%E2%80%99s_Wife_by_John_Bridges.jpg

29 Καὶ εὐθὺς ἐκ τῆς συναγωγῆς ἐξελθόντες ἦλθον εἰς τὴν οἰκίαν Σίμωνος καὶ Ἀνδρέου μετὰ Ἰακώβου καὶ Ἰωάνου. 30 ἡ δὲ πενθερὰ Σίμωνος κατέκειτο πυρέσσουσα, καὶ εὐθὺς λέγουσιν αὐτῷ περὶ αὐτῆς. 31 καὶ προσελθὼν ἤγειρεν αὐτὴν κρατήσας τῆς χειρός· καὶ ἀφῆκεν αὐτὴν ὁ πυρετός, καὶ διηκόνει αὐτοῖς. 

32 Ὀψίας δὲ γενομένης, ὅτε ἔδυσεν ὁ ἥλιος, ἔφερον πρὸς αὐτὸν πάντας τοὺς κακῶς ἔχοντας καὶ τοὺς δαιμονιζομένους· 33 καὶ ἦν ὅλη ἡ πόλις ἐπισυνηγμένη πρὸς τὴν θύραν. 34 καὶ ἐθεράπευσεν πολλοὺς κακῶς ἔχοντας ποικίλαις νόσοις, καὶ δαιμόνια πολλὰ ἐξέβαλεν, καὶ οὐκ ἤφιεν λαλεῖν τὰ δαιμόνια, ὅτι ᾔδεισαν αὐτόν. 

\fig{01-35} % https://commons.wikimedia.org/wiki/File:Brooklyn_Museum_-_Christ_Retreats_to_the_Mountain_at_Night_(J%C3%A9sus_se_retira_la_nuit_sur_la_montagne)_-_James_Tissot.jpg

35 Καὶ πρωῒ ἔννυχα λίαν ἀναστὰς ἐξῆλθεν καὶ ἀπῆλθεν εἰς ἔρημον τόπον, κἀκεῖ προσηύχετο. 36 καὶ κατεδίωξεν αὐτὸν Σίμων καὶ οἱ μετ’ αὐτοῦ, 37 καὶ εὗρον αὐτὸν καὶ λέγουσιν αὐτῷ ὅτι Πάντες ζητοῦσίν σε. 38 καὶ λέγει αὐτοῖς· Ἄγωμεν ἀλλαχοῦ εἰς τὰς ἐχομένας κωμοπόλεις, ἵνα καὶ ἐκεῖ κηρύξω· εἰς τοῦτο γὰρ ἐξῆλθον. 39 καὶ ἦλθεν κηρύσσων εἰς τὰς συναγωγὰς αὐτῶν εἰς ὅλην τὴν Γαλιλαίαν καὶ τὰ δαιμόνια ἐκβάλλων.

\fig{01-40} % https://commons.wikimedia.org/wiki/File:Iscelenie_prakajennogo.jpg

	40 Καὶ ἔρχεται πρὸς αὐτὸν λεπρὸς παρακαλῶν αὐτὸν καὶ γονυπετῶν λέγων αὐτῷ ὅτι Ἐὰν θέλῃς δύνασαί με καθαρίσαι. 

41 καὶ σπλαγχνισθεὶς ἐκτείνας τὴν χεῖρα αὐτοῦ ἥψατο καὶ λέγει αὐτῷ Θέλω, καθαρίσθητι. 42 καὶ εὐθὺς ἀπῆλθεν ἀπ’ αὐτοῦ ἡ λέπρα, καὶ ἐκαθερίσθη. 43 καὶ ἐμβριμησάμενος αὐτῷ εὐθὺς ἐξέβαλεν αὐτόν, 44 καὶ λέγει αὐτῷ Ὅρα μηδενὶ μηδὲν εἴπῃς, ἀλλὰ ὕπαγε σεαυτὸν δεῖξον τῷ ἱερεῖ καὶ προσένεγκε περὶ τοῦ καθαρισμοῦ σου ἃ προσέταξεν Μωϋσῆς εἰς μαρτύριον αὐτοῖς. 

45 ὁ δὲ ἐξελθὼν ἤρξατο κηρύσσειν πολλὰ καὶ διαφημίζειν τὸν λόγον, ὥστε μηκέτι αὐτὸν δύνασθαι φανερῶς εἰς πόλιν εἰσελθεῖν, ἀλλ’ ἔξω ἐπ’ ἐρήμοις τόποις ἦν· καὶ ἤρχοντο πρὸς αὐτὸν πάντοθεν.



\sloppypar


\end{document}
